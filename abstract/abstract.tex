\documentclass[12pt]{article}
\usepackage[english]{babel}
\usepackage[utf8]{inputenc}
\usepackage[T1]{fontenc}
\usepackage[margin=2.54cm]{geometry}
\usepackage{setspace}

\begin{document}

\begin{center}
  \Large \textbf{\textsf{Musical contour in composition}}
\end{center}

\thispagestyle{empty}

\doublespacing

Contour is the shape or format of objects. In Music, they can be
associated to elements like pitch, density, rhythm, and can represent
a parameter in function of another, such as pitch in function of time.
Contours are important because, as well as motives and pitch sets,
contours help to give coherence to a composition.

Theories of contours have been used in areas such as Ear Training and
Analysis, but the systematic use of contours for generation of
compositional material is an issue still lacking literature.

In this paper we present the partial results of our research on the
use of contour in musical compositions. We present the contour
processing software that we are developing, and contour usage examples
in a woodwind quintet composed by first author of this paper in his
master's research.

\end{document}
